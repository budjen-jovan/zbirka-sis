\begin{slikaDesno}{fig/rect-per15.pdf}
    \textbf{\ID.}\label{ID:rect_pulse_train_FS} 
    Дат jе напонски сигнал $v = v(t)$ облика периодичне поворке униполарних правоугаоних 
    импулса амплитуде ${V_{\rm m} = 5\unit{V}}$, као на слици. Трајање импулса је $DT$ где је 
$D = 25\%$ (тзв. \textit{фактор испуне}), а учестаност jе $f = 1\unit{kHz}$. 
Одредити развоj овог сигнала у комплексан Фуриjеов ред, $V[k]$, на основном периоду $T$.
\end{slikaDesno} \\

\textsc{\underline{Решење}}: Развој у Фуријеов ред се може потражити по дефиницији применом 
аналитичке релације 
$\DS V[k] = \int_{\langle T \rangle} v(t) \ee^{-\jj k \upomega_{\rm F} t} \, \de t$, 
поступком\footnote{Користи се резултат $\int e^{kx} \, \de x = \frac{1}{k} e^x + C$}
\begin{align}
    V[k] = \int_{0}^T v(t) \ee^{-\jj k \upomega_{\rm 0} t} \, \de t 
         = \int_{0}^{DT}  V_{\rm m} \ee^{-\jj k \frac{2\uppi}{T} t} \, \de t
         = -\dfrac{V_{\rm m}}{\jj k \frac{2\uppi}{T}} 
         \ee^{-\jj k \frac{2\uppi}{T} t}\bigg|_{t = 0}^{t = DT}
         = V_{\rm m}\dfrac{
            1
            -
            \ee^{-\jj k 2\uppi D}
         }{\jj k \frac{2\uppi}{T}} 
\end{align}
Добијени облик може се поједноставити примедбом 
$\sin(x) = \dfrac{\ee^{\jj x} - \ee^{-\jj x}}{\jj 2} = \ee^{\jj x} \dfrac{1 - \ee^{-\jj 2x}}{\jj 2}$,
односно, 
$\dfrac{1 - \ee^{-\jj 2x}}{\jj 2} = \ee^{-\jj x} \sin(x) $,
одакле се може писати
\begin{align}
    V[k] = V_{\rm m}DT 
     \underbrace{
     \dfrac{
        1
        -
        \ee^{-\jj 2 (k \uppi D) }
     }{\jj 2 }}_{ = \ee^{-\jj k \uppi D/2} \sin(k\uppi D) }
     \cdot 
     \dfrac{1}{k \uppi D} 
     = V_{\rm m} D T \dfrac{\sin(k\uppi D)}{k\uppi D} \ee^{-\jj k \uppi D/2} 
     = V_{\rm m} D T \sinc(k \uppi D) \ee^{-\jj k \uppi D/2}.
\end{align}
Добијени резултат може се одредити и таблично којом приликом ће се члан 
$\ee^{-\jj k \uppi D/2}$ појавити услед кашењења у времену периодичне поворке правоугаоних 
импулса симетричне око ординате. 
