\PID
Каузални систем описан је диференцном једначином 
\begin{equation}
    y[n] - \sqrt 3 y[n-1] + y[n-2] = x[n],
\end{equation}
где су познати помоћни услови 
$y[-1] = 1$ и $y[-2] = 0$. 
(а) Испитати \myul{асимптотску} стабичност датог система. 
(б) Одредити сопствени оздив, $y_{\rm a}[n]$, датог система 
за задате помоћне услове. 
(в) Одредити принудни одзив, $y_{\rm f}^{\text{(в)}}[n]$, датог система
за побуду $x[n] = \updelta[n]$.
(г) Одредити принудни одзив $y_{\rm f}^{\text{(г)}}[n]$ датог система
за побуду $x[n] = \cos(n\uppi) \uu[n]$.
\\[2mm]

\textsc{\underline{Решење:}}
Дата диференцна једначина се може записати у операторском облику 
као $P(\EE) y[n] = Q(\EE) x[n]$, где су $P(\EE) = 1 - \sqrt 3 \EE + \EE^2$
и $Q(\EE) = \EE^2$,
где је $\EE = n \mapsto n+1$ дискретни оператор естимације. 

(а) Асимптотска стабилност ситема одређује се испитивањем 
структуре корена карактеристичног полинома $P(\EE)$. 
Пошто дати систем има корене карактеристичног полинома 
\begin{equation}
    P(\uplambda) = 0 
    \Rightarrow
    \uplambda = 
    \dfrac{\sqrt 3 \pm \sqrt{3 - 4}}{2} = 
    \dfrac{\sqrt 3 \pm \jj}{2} = 
    \exp \left( \pm \jj \dfrac{\uppi}{6} \right),
\end{equation}
односно они представљају комплексно конјуговани пар 
$\uprho \ee^{\pm \jj \upphi}$, где су 
$\uprho = 1$, $\upphi = \uppi/6$.
Пошто постоје само једноструки корени на јединичној кружници комплексне 
равни, систем је гранично стабилан. 

(б) Према резултату додатка 
\ref{dod:diferencne}, на основу добијених корена карактеристичног 
полинома. имају се каркатеирстичне функције из скупа 
$\{ \cos(n\uppi/6), \sin(n\uppi/6) \}$, па је хомогено решење одзива 
у облику $y[n] = \cancelto{1}{\uprho^n} \left( A \cos(n\uppi/6) + B \sin(n\uppi/6) \right)$,
а коефицијенти $A$ и $B$ се одређују из датих помоћних услова: 
\begin{align}
    y[-1] = 1 &= \cos(-\uppi/6) A + \sin(-\uppi/6) B &= 
    \dfrac{\sqrt 3}{2} A - \dfrac{1}{2} B, \\
    y[-2] = 0 &= \cos(-2\uppi/6) A + \sin(-2\uppi/6) B &=
    \dfrac{1}{2} A - \dfrac{\sqrt 3}{2} B.  
\end{align}
Решавањем добијеног система једначина налазе се непознати коефицијенти
$A = \sqrt 3$, $B = 1$, па је тражени сопствени одзив у облику
\begin{equation}
    y_{\rm a}[n] =
    \sqrt 3 \cos\left(\dfrac{n\uppi}{6}\right) +
    \sin\left(\dfrac{n\uppi}{6}\right) , \qquad
    n \geq -2.
\end{equation}

(в) Импулсни одизв се такође тражи у хомогеном облику за неки други 
избор коефицијената испред карактеристичних функција, под 
претпоставком нултих преиницијалних услова $y[n<0] = 0$, као
\begin{equation}
    h[n] = \cancelto{1}{\uprho^n} \left( C \cos(n\uppi/6) + D \sin(n\uppi/6) \right)
    \label{\ID.eq1}
\end{equation}

Том приликом, коефицијенти се могу пронаћи рекурзивним израчунавањем 
вредности одзива за задату побуду, преуређивањем дате диференцне једначине
$y[n] = x[n] + \sqrt 3 y[n-1] - y[n-2]$, препознајући да је 
$y[n<0] = 0$, за побуду у облику $x[n] = \updelta[n]$, на основу поступка: 
\begin{align}
    y[0] &= \updelta[0] + \cancelto{0}{\sqrt 3 y[-1]} - \cancelto{0}{y[-2]} 
         = \updelta[0] = 1, \qquad \label{\ID.eqsub1}
          &\text{замена у наредни корак} \hookleftarrow \\
    y[1] &= \cancelto{0}{\updelta[1]} + \sqrt 3 
    \cancelto{1}{y[0]} - \cancelto{0}{y[-1]} 
         = \sqrt 3, \qquad
          &\text{замена у наредни корак}  \hookleftarrow \\ 
    y[2] &= \cancelto{0}{\updelta[2]} + \sqrt 3 
    \cancelto{\sqrt 3}{y[1]} - \cancelto{1}{y[0]} 
         = \sqrt 3 - 1, \qquad
          &\text{замена у наредни корак}  \hookleftarrow \\          
\end{align}
Примећујемо да оваквим итеративним поступком можемо одредити 
одзив $y[m]$ у произвољном тренутку $m$, након $m$ израчунавања. Међутим, 
будући да познајемо аналитички облик решења као \ref{\ID.eq1}
онда су нам ова два резултата довољни за израчунавање коефицијената. 
\begin{align}
    h[1] &= C \cos(\uppi/6) + D \sin(\uppi/6) =
    \dfrac{\sqrt{3}}{2} C + \dfrac{1}{2} D = \sqrt 3, \\
    h[2] &= C \cos(2\uppi/6) + D\sin(\uppi/6) =
    \dfrac{1}{2} C + \dfrac{\sqrt{3}}{2} D = \sqrt 3 - 1.
\end{align}
одакле се налазе коефицијенти $C = 4 - \sqrt 3$, $D = 3 - 2\sqrt 3$, па се одзив 
на задату побуду може изразити у облику
\begin{equation}
    y_{\rm f}^{\text{(в)}}[n] = 
    \left(4 - \sqrt 3\right)
    \cos\left(\dfrac{n\uppi}{6}\right) + 
    \left(3 - 2\sqrt 3\right)
    \sin\left(\dfrac{n\uppi}{6}\right) , \qquad
    n \geq 1.
\end{equation}
Важно је нагласити да дати израз има смисла само у наведеном опсегу, ван њега, 
тај резултат строго нема смисла. На пример, за $n = 0$ је према њему 
$y_{\rm f}^{\text{(в)}}[n=0] = 4 - \sqrt{3}$ док на основу 
\eqref{\ID.eqsub1} треба да буде $y_{\rm f}^{\text{(в)}}[n=0] = 1$. Ова несагласност 
потиче од чињенице да је \textit{фитовање} коефицијената $C$ и $D$ оправдано само у случају 
хомогене диференцне једначине, што строго важи за $n > 0$ јер је $x[n=0]\neq0$, у 
овом конкретном случају. 

(г) Тачка се оставља читаоцу за вежбу. Принудни одзив за побуду $x[n] = \cos(n\uppi) \updelta[n]$ одређује се уважавањем 
партикуларног дела који се одређује на основу израза 
$y_{\rm p}[n] = \Re{ \dfrac{\ee^{\jj \uppi n}}{P(\jj\uppi)} }$.  
