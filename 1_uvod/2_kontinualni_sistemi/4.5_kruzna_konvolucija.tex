\begin{slikaDesno}{fig/rect-per.pdf}
\PID За 
периодичне сигнале са основним 
периодом $T$ може се дефинисати 
операција \textit{периодичне
конволуције} као 
$$x \circledast y = \int_{0}^T 
x(\uptau) y(t-\uptau) \, \de \uptau.$$
\end{slikaDesno} \\
Периодична поворка 
униполарних
правоугаоних импулса једнаког трајања импулса и паузе,
$v = v(t)$, приказана је на слици. Параметре $A$ и $Т$ сматрати познатим.
Одредити $v \circledast v$.
\vspace*{2mm}

\textsc{\underline{Резултат:}}
Тражени израз је $v \circledast v = 
\dfrac{A^2 T}{2}
\sum_{k = -\infty}^{\infty}  
{\rm tri}\left( 
\dfrac{2(t - kT)}{T} - 1
\right)
$.