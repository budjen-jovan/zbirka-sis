\PID \label{z:bibo_imp}
Нека је познат импулсни одзив $h(t)$ неког линеарног временски инваријантног система.
У зависности од тог импулсног одзива, дискутовати стабилност тог система у BIBO смислу.
\vspace*{2mm}

\textsc{\underline{Решење}}: Одзив LTI система на произвољну побуду $x(t)$ може се изразити 
помоћу конволуције као 
$y(t) = x(t) \ast h(t) = \int_{-\infty}^{\infty} x(\uptau) h(t - \uptau )\, \de \uptau$. Претпоставимо да је 
побудни сигнал апсолутно ограничен као $|x(t)| \leq B_{x}$, за неко ограничење $B_x$, онда се има 
\begin{equation}
    |y(t)| = \int_{-\infty}^{\infty} |x(\uptau)| |h(t - \uptau )|\, \de \uptau \leq 
    B_{\rm x} \int_{-\infty}^{\infty} |h(t - \uptau )|\, \de \uptau = B_{y}.
\end{equation}
Одатле се има да је потребан и довољан услов да је и одзив ограничен $B_y < \infty$, дат у облику
\begin{equation}
\int_{-\infty}^{\infty} |h(\uptau )|\, \de \uptau < \infty. \label{eq:\ID.1}
\end{equation}

Коначно, LTI систем  чији је импулсни одзив $h(t)$ је стабилан у BIBO смислу ако и само ако је
импулсни одзив апсолутно интеграбилан, односно ако важи услов \eqref{eq:\ID.1}.

