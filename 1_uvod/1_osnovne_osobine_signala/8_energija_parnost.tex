\noindent
\textbf{\ID}. \label{z:zbirP}
Познато је да су енергије реалних 
сигнала $x = x(t)$ и $y = y(t)$, $W_x$ и $W_y$ редом, коначне. 
Одредити (а) услов, који треба да задовољавају сигнали $x$ и $y$, под 
којим је снага сигнала $z(t) = x(t) + y(t)$ једнака 
$W_z = W_x + W_y$. На основу резултата из претходне тачке 
(б) доказати једнакост:
$$
W\{x\} =  W\{ {\rm Ev}\{x\} \} + 
W\{ {\rm Od}\{x\} \},
$$
где $W\{x\}$ означава енергију сигнала $x$.
\\[2mm]

\textsc{\underline{Решење}}:
(а) Снага сигнала $z(t) = x(t) + y(t)$ може се изразити као 
\begin{equation}
    \underbrace{ \int_{-\infty}^{\infty} (x(t) + y(t))^2 \, {\rm d}t }_{W_z}
    = \underbrace{ \int_{-\infty}^{\infty} x(t)^2 \, {\rm d}t }_{W_x} +
    \underbrace{ \int_{-\infty}^{\infty} y(t)^2 \, {\rm d}t }_{W_y} +
    2 \int_{-\infty}^{\infty} x(t)y(t) \, {\rm d}t.
    \label{eq:\ID.1}
\end{equation}
Да би се остварио услов $W_z = W_x + W_y$, потребно је да буде 
$\displaystyle \int_{-\infty}^{\infty} x(t)y(t) \, {\rm d}t = 0$.

(б) На основу резултата претходне тачке, једнакост коју треба доказати је тачна, 
будући да је услов интеграл непарне функције у симетричним границама,  
$\displaystyle    
    \int_{-\infty}^{\infty} \underbrace{ {\rm Ev}\{x\} \cdot {\rm Od}\{x\} }_{\text{Непарна фукција}}   \, {\rm d}t = 0.
    \label{eq:\ID.2}
$