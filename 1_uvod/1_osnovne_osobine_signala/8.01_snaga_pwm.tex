\begin{slikaDesno}{fig/rect-pwm.pdf}
    \PID \label{z:snaga_pwm}
    Дат jе напонски сигнал $v = v(t)$ облика периодичне поворке униполарних правоугаоних 
    импулса амплитуде ${V_{\rm m} = 5\unit{V}}$, као на слици. Трајање импулса је $DT$ где је 
    $0 \leq D \leq 1$ (тзв. \textit{фактор испуне}), а учестаност сигнала је $f$. Одредити средњу снагу 
    наизменичне компоненте тог сигнала у функцији фактура испуне $P_{\rm AC} = P_{\rm AC}(D)$. 
\end{slikaDesno}
\\[2mm]

\textsc{\underline{Решење:}}
На основу става доказаног у задатку \ref{z:energija_ac_dc} средња снага комплетног сигнала
једнака је збиру средњих снага наизменичне и сталне компонентe, $P = P_{\rm AC} + P_{\rm DC}$. 
Стална компонента сигнала једнака је 
$V = \dfrac{1}{T} \int_0^T v(t) \de t = \dfrac{1}{T} \int_0^{DT} V_{\rm m} \de t = DV_{\rm m}$, па је 
снага сталне компоненте $P_{\rm DC} = D^2 V_{\rm m}^2$. 

\begin{slikaDesno}{fig/pwm_snaga.pdf}
    Средња снага комплетног сигнала је 
    $P = \dfrac{1}{T} \int_0^T v^2(t) \de t = \dfrac{1}{T} \int_0^{DT} V_{\rm m}^2 \de t = DV_{\rm m}^2$,
    одакле је средња снага наизменичне компоненте   
    $P_{\rm AC} = V_{\rm m}^2 D(1 - D)$, што представља параболу по $D$, као што је илустровано на слици \ID.2.
\end{slikaDesno}
