\subsubsection{\textit{Периодичност и енергија сигнала}}
\PID 
\noindent 
Утврдити да ли су следећи сигнали периодични и за оне који то јесу израчунати 
основни период:
\begin{multicols}{2}
\begin{enumerate}
\item[(а)] $x(t) = \cos(3t) + \sin(5t)$;
\item[(б)] $x(t) = \cos(6t) + \sin(\uppi t)$;
\item[(в)] $x(t) = \cos(6t) + \sin(8t) + {\rm e}^{{\rm j}2t}$.
\end{enumerate}
\end{multicols}

\textsc{\underline{Решење}}:
Периодичност збира континуалних сигнала $f_1(t)$, $f_2(t)$, $\ldots$, и $f_n(t)$ може се дискутовати на основу њихових основних 
периода $T_1$, $T_2$, $\ldots$, и $T_n$. Претпоставимо да је основни период сигнала 
$f(t) = f_1(t) + f_2(t) + \ldots + f_n(t)$ једнак $T$. Тада је јасно да се период сваког од сигнала сабирака 
мора садржати цео број пута у сигналу збира, односно $T = n_1T_1 = n_2T_2 = \ldots = n_nT_n$, где су $n_1$, $n_2$, $\ldots$, и $n_n$ 
цели бројеви. Пошто је основни период најмањи такав период, то значи да је $T$ најмањи број који се цео број пута 
садржи у сваком од периода сигнала сабирака, односно је
\begin{equation}
    T = {\rm NZS} \{ T_1, T_2, \ldots, T_n\}.
    \label{eq:nzs}
\end{equation}
Такав резултат ће постојати уколико је $\dfrac{T_i}{T_j} = \dfrac{n_i}{n_j} \in \mathbb Q, \forall i,j$,
односно, ако је однос сваког пара периода рационалан број.    
За такве периоде кажемо да су \textit{рационално самерљиви}.

\begin{enumerate}
    \item[(а)] Сигнал $x(t) = \cos(3t) + \sin(5t)$ је периодичан, јер је основни периоди\footnote{
    Основни период синусоиде $\sin(\upomega_0 t + \uptheta)$ је $T = 2\uppi /\upomega_0$.
    } сабирака рационално самерљиви, односно,  
    $T = {\rm NZS} \left\{ \dfrac{2\uppi}{3}, \dfrac{2\uppi}{5} \right\} = \dfrac{2\uppi}{15}$. 
    \item[(б)] Сигнал $x(t) = \cos(6t) + \sin(\uppi t)$ је 
    апериодичан јер периоди сабирака, $\dfrac{2\uppi}{6}$ и $2$, нису рационално самерљиви пошто је 
    $\uppi \not\in \mathbb Q$.
    \item [(в)] Сигнал $x(t) = \cos(6t) + \sin(8t) + {\rm e}^{{\rm j}2t}$ је периодичан, 
    јер су основни периоди сабирака
    $\dfrac{\uppi}{3}$, $\dfrac{\uppi}{4}$, и $\uppi$ рационално самерљиви, а период
    је $T = {\rm NZS} \left\{ \dfrac{\uppi}{3}, \dfrac{\uppi}{4}, \uppi \right\} = \uppi$.
\end{enumerate}