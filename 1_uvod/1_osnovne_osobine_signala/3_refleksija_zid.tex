
\begin{slikaDesno}[.833]{fig/meh.pdf}
\PID 
На слици је приказано круто тело масе $m$ 
које може да се креће по подлози без трења. 
Брзина тела дата је као ${\bf v} = v(t) \, 
{\bf i}_x$. У тренутку $t_0 = 0$ блок се 
апсолутно еластично судара са 
непокретним зидом након чега 
се креће брзином алгебарског интензитета 
$v(t) = -v_0$. (а) Одредити и 
изразити $v(t)$ за $-\infty< t < \infty$. 
(б) Одредити и нацртати временски дијаграм
алгебарског интензитета  
силе којом зид делује на блок ${\bf N} = N(t) 
{\bf i}_x$.
\end{slikaDesno}
\vspace*{5mm}

\begin{slikaDesno}[1]{fig/blok_zid.pdf}
\hspace*{5mm}
\textsc{\underline{Резултат}}:
(а) $v(t) = v_0 \bigl(1 - 2\uu(t)\bigr)$.
(б) Тражени дијаграм приказан је на слици \ID.2.
Временски облик силе нормалне реакције зида дат је у облику 
$N(t) = -2mv_0 \, \updelta(t)$. Налгасимо да је у овом случају димензија
мере Делта импулса механички импулс (количина кретања). Односно, може се рећи да механички импулс, 
који је тело примило приликом краткотрајног дејства силе, одговара мера Делта импулса силе која је на 
њега том приликом деловала.
\end{slikaDesno}