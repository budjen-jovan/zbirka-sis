%\vspace{-10mm}
\subsubsection{\textit{Делта импулс и Хевисајдова одскочна функција}}
\noindent
\PID 
Скицирати временски дијаграм сигнала датог изразом $x(t) = \updelta(f(t))$ где је 
$f(t) = t^2 - 1$, а
$\updelta(t)$ Дираков импулс. \\[2mm]

\textsc{\underline{Решење}}: 

\begin{slikaDesno}
[1]
[$t^2 - 1$ (- - -), $\updelta(t^2 - 1)$ (---)]
{fig/dirac_pm1.pdf}
Користећи дефинициону особину Дираковог импулса да је $\updelta(t \neq 0) = 0$, 
закључујемо да је $x(t) = 0$ за $f(t) \neq 0$. На основу тога, ненулту вредност ће дати сигнал имати 
само у тачкама које су решење једначине $f(t) = t^2 - 1 = 0$ односно $t = \pm 1$.  
Даље, приметимо да је за даљу анализу релевантна само непосредна околина 
тачака $t = \pm 1$, односно, једино је релевантно колико „брзо“ сигнал пролази
кроз нулу -- за шта је мера први извод у тој тачки. Одређивањем извода  има се да је 
$\dfrac{\de f(t)}{\de t} = 2t = \pm 2$ за $t = \pm 1$. 
\end{slikaDesno}
Користећи особину да је $\updelta( k(t - \uptau) ) = \dfrac{1}{|k|} \updelta(t - \uptau)$, онда 
се може писати 
${x(t) = \dfrac12 \updelta(t - 1) + \dfrac12 \updelta(t + 1)}$. Добијени резулат илустрован је на слици 
\ID.1. 