%\vspace{-10mm}
\subsubsection{\textit{Делта импулс и Хевисајдова одскочна функција}}
\noindent
\PID 
Скицирати временски дијаграм сигнала датог изразом $x(t) = \updelta(f(t))$ где је 
$f(t) = t^2 - 1$, а
$\updelta(t)$ Дираков делта импулс. \\[2mm]

\textsc{\underline{Решење}}: 
Строга анализа проблема композиције делта импулса са датом функцијом, $\updelta(t) \circ f(t) = \updelta(f(t))$ 
спроводи се у математичкој теорији дистрибуција. Проблем ћемо анализирати са инжењерског становишта, 
тумачећи га на следећи начин. Пре свега, уочимо да је за $f(t) \neq 0$ тада $\updelta(f(t)) = 0$, по дефиницији.    
Са друге стране, тачке $f(t) = 0$ представљају нуле функције $f(t)$. Уочимо једну нулу те функције, $t_0$, и распишимо
функцију $f(t)$ у Тејлоров ред око тачке $t_0$:
\begin{eqnarray}
    f(t) = \cancelto{0}{f(t_0)} + f'(t_0) (t-t_0) + \dfrac{f''(t_0)}{2} (t-t_0)^2 + \cdots
\end{eqnarray}
онда приметимо да нас интересује резултат само у веома непосредној околини тачке $t_0$, те можемо занемарити све
чланове реда који су већи од првог, чиме преостаје $f(t) \approx f'(t_0) (t-t_0)$. Одавде у околини тачке 
$t_0$ вреди: 
\begin{equation}
    \updelta(f(t)) = \updelta(f'(t_0) (t-t_0)) = \dfrac{\updelta(t-t_0)}{|f'(t_0)|},
\end{equation}
при чему је искоришћено својство скалирања аргумента делта импулса\footnote{Својство скалирања аргумента делта импулса:
$\updelta(at) = \dfrac{1}{|a|} \updelta(t)$.}.
Закључујемо да се у тачки $t_0$, таквој да је $f(t_0) = 0$, дешава појава делта импулса чија је мера $1/|f'(t_0)|$.
Пошто се по један импулс налази у свакој тачки нуле функције $f(t)$ коначно се може писати:
\begin{equation}
    \updelta(f(t)) = \sum_{\substack{t_i \\[0.5mm] f(t_i)=0}} \dfrac{\updelta(t-t_i)}{|f'(t_i)|},
\end{equation}

\begin{slikaDesno}{fig/dirac_pm1.pdf}
У конкретном датом случају, нуле функције $f(t)$ су ${t_1 = -1}$ и {$t_2 = 1$}, па је:
\begin{eqnarray}
    \updelta(f(t)) = \dfrac{\updelta(t+1)}{2} + \dfrac{\updelta(t-1)}{2},
\end{eqnarray}
што је и тражени резултат који је приказан на слици \ID.1.
Функција $f(t)$ је уцртана сивом бојом, а сигнал $\updelta(f(t))$ црном бојом.

\end{slikaDesno}